\section{Giới thiệu}
\subsection{Bài toán nhận diện hành động qua video}
%TODO: chém gió qua về AI trong đời sống.
Sự phát triển của nhân loại thời đại 4.0 đang đến giai đoạn mạnh mẽ, đặt
ra những nhu cầu tất yếu mà ở các thời kì trước không có, công nghệ hiện đại
là một trong những nhu cầu quan trọng đó. Ngoài hiện đại về chất liệu sản xuất, khả năng tương tác cao với
con người đang là xu thế cho sự ra đời của hàng loạt sản phẩm công nghệ, cụ thể là các sản phẩm trí tuệ nhân tạo (AI).

AI đã xuất hiện khá lâu nhưng thay vì chỉ mới ở mức độ lý thuyết cơ bản như thời ban đầu, thì nay các ứng dụng
thực tiễn của AI đã quá phổ biến để có thể thay đổi cuộc sống con người. Ngày nay, AI xuất hiện hầu hết trong tất cả các lĩnh
vực từ Giao thông, Y tế, Ngân hàng cho tới Giải trí và rất nhiều lĩnh vực khác. Các ứng dụng trí tuệ nhân tạo đang
dần thay thế các hoạt động thủ công của con người bằng sự tự động của các thiết bị, phần mềm.

Hiểu đơn giản, AI là ngành khoa học về máy tính với mục tiêu giúp máy móc có khả năng hoạt động thông minh như con người trong phạm vi nào đó.
Từ đó các máy móc có thể  thực hiện được những công việc từ đơn giản tới phức tạp, giúp làm giảm khối lượng công việc cho con người.
Tuy nhiên, trên thực tế khả năng của AI đa phần chỉ mới tiệm cận khả năng thực sự của con người, thậm chí đang ở một số bài toán, AI vẫn đang ở giai đoạn
nghiên cứu phát triển. Mục tiêu của người nghiên cứu AI, là đưa khả năng đó lên mức cao hơn, đảm bảo tính khả thi ứng dụng của các sản phẩm AI được tạo ra.

%TODO: nhận diện hành động là gì?
Các ứng dụng phổ  biến của AI có thể kể đến là xử lý ảnh và đưa ra dự đoán về bệnh trong Y tế, nhận diện vật thể trong Giám sát an ninh,
hay xác định tình trạng giao thông trong thành phố , đó chủ yếu là các bài toán xử lý trên hình ảnh. Ngoài các bài toán xử lý trên ảnh,
việc xử lý dữ liệu dạng chuỗi như video cũng mang lại ứng dụng quan trọng như nhận diện hành vi con người để  máy móc có khả năng
hiểu được ý nghĩa của các hành vi nhằm tương tác tốt hơn với con người. Các ứng dụng  nổi bật  của việc nhận diện hành vi con người có thể
kể đến như phát hiện hành vi bất thường (móc túi, trộm cắp), nhận diện và dự đoán hành vi con người trong thời gian gần (nhận diện con người băng qua đường cho xe tự hành),
hay giúp máy móc tương tác với hành vi con người (điều khiển smart TV bằng cử chỉ).

%TODO: challanges trong bài toán này?
Dữ liệu dạng chuỗi này có nhiều ưu điểm cũng như thách thức khác hẳn so với xử lý hình ảnh thông thường, với đặc tính thời gian, lượng dữ liệu lớn hơn,
cần có những phương pháp đặc thù để thao tác hiệu quả. Tuy nhiên các phương pháp hiện nay về xử lý dữ liệu dạng chuỗi vẫn còn một số vấn đề,
có những phương pháp độ chính xác chưa tốt, hay cũng có những phương pháp tuy độ chính xác khá cao nhưng tính ứng dụng chưa cao do chi phí bỏ ra là quá lớn.

Từ những ứng dụng đã được kể  đến cũng như thách thức đang gặp phải hiện nay của bài toán mà nhóm em đã quyết định chọn đề tài
\textit{Nhận diện hành vi qua video} để nghiên cứu cho giai đoạn Đề cương luận văn cũng như giai đoạn luận văn sau này.
%TODO: mục tiêu chung của đề  tài này là gì?
Mục tiêu của Đề cương luận văn này là tìm và nghiên cứu phương pháp cải thiện tính khả thi của các pháp sẵn có, cũng nhưng cải tiến mô hình để cho khả năng
nhận diện hành vi chính xác hơn cho mô hình cải tiến.




\subsection{Mục tiêu của giai đoạn đề cương luận văn}
%TODO: giai đoạn đề cương luận văn đề ra mục tiêu gì? servey, implements các pp đã có, so sánh, kết quả bước đầu.
Trong giai đoạn đề cương luận văn này, mục tiêu của nhóm là khảo sát, nghiên cứu, phát triển hệ thống nhận diện hành vi qua video,
cụ thể  như sau:
\begin{itemize}
    \item Thực hiện khảo sát những phương pháp nổi bật trong lĩnh vực nghiên cứu, từ thời điểm hiện tại trở về trước. Nghiên
          cứu ưu nhược điểm của từng phương pháp cũng như tiến hành hiện thực một số phương pháp nếu được để nghiệm thu và đánh giá kết quả.
    \item Từ kết quả khảo sát chung, tiến hành đánh giá, so sánh để chọn ra phương pháp khả thi nhất làm phương pháp nền tảng để đi sâu nghiên cứu.
    \item Tìm ra những vấn đề tồn đọng của phương pháp đã chọn, nghiên cứu cải tiến để được mục đích mong muốn.
    \item Sưu tập dữ liệu và các tài nguyên cần thiết cho phương pháp đã chọn, cũng như thử hiện thực phương pháp để có kết quả bước đầu,
          nhằm đảm bảo tính khả thi của phương pháp, từ đó tạo tiền đề cho việc đi sâu phát triển ở giai đoạn về sau.
    \item Tìm ra các vấn đề tồn đọng chưa được giải quyết và đề xuất các hướng phát triển sau này.
\end{itemize}

Để phù hợp với thời gian  hạn chế của giai đoạn đề cương, nhóm tập trung nghiên cứu bài toán trong phạm vi cụ thể :
\begin{itemize}
    \item Đối tượng để nhận diện hành vi là con người (human).
    \item Hiện tại dữ liệu đầu vào đang là dạng skeleton data, sau này sẽ là dữ liệu video RGB, việc chuyển từ RGB sang skeleton hiện
          đã có các phương pháp hoạt động tốt.
    \item Chỉ mới thao tác trên các hành động phổ biến, gồm 60 hành động hay là 60 class cho bài toán nhận diện hành vi qua video, về sau nhóm sẽ mở rộng
          hệ thống cho nhiều class hơn.
    \item Số đối tượng là 1 người độc lập, sau này sẽ là nhận diện trên một nhóm người cùng với tương tác giữa các đối tượng (nhận diện ở mức độ cao cấp hơn).
\end{itemize}

\subsection{Cấu trúc của đề cương luận văn}
Đề cương luận văn được tổ chức thành các  phần như sau:


\begin{enumerate}[label=52.,start=2]
    \item  hahasdf
    \item asfdasdfjl

\end{enumerate}


